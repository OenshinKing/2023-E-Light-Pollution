\section{\textbf{Light Pollution Risk Level Metrics}}

\subsection{\textbf{Factors on Light Pollution Problem}}
Light pollution attributes to ALAN and impacts humans and environment. Researches have indicated the strong correlation between ALAN and the social-economic level. Here we detect four factors on light pollution for our further discussion.\par

\subsubsection{Light Sources}
There are diverse light emission sources that can be classified into two types according to characteristics of emitted light: (1)Stationary light emission sources such as lights outside houses, offices, shops, parking lots. (2)Non-stationary sources such as headlights in vehicles.\par
What's more, sources of light are mostly powered by electricity with various devices. Conventional types of devices include fluorescent lights, incandescent lights, natrium gas lights. Recent years those sources are being replaced by light-emitted diodes (LEDs) that are more economic and have better performance, while LEDs may have disadvantages in terms of color and sharp brightness that may cause discomfort glare and annoyance\textsuperscript{\cite{ref5}}. \par

\subsubsection{Society}
Nighttime light data from satellites is the reflection of not only light pollution but also social activities. Areas with higher level of urbanization exhibit brighter ALAN due to the concentration of commercial activities, public facilities and nightlife. In essence, the intensity and distribution of ALAN are a direct consequence of the diverse social activities that occur across different regions. These activities not only shape the patterns of light emissions but also highlight the extent of human presence and development. Therefore, the data collected from nighttime lights can be seen as a proxy for understanding the spatial and temporal dynamics of human activities on a global scale.\par 

\subsubsection{Human}
The human here is detected from society as the factor that is impacted by light pollution. Generally speaking, light pollution affects human health both physically and mentally\textsuperscript{\cite{ref3}}. Previous researches found that the negative effects of ALAN consist of direct parts and indirect parts. In terms of indirect effects, light exposure would affect the duration and quantity of sleep and then impact physical and mental health. When it comes to direct effects, light will disrupt normal circadian systems of humans. In summary, we humans will suffer a variety of complex impacts of light pollution.\par 


\subsubsection{Environment}


\subsection{\textbf{Light Pollution Index}}
