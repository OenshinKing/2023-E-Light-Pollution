\begin{thebibliography}{99}
    \bibitem{ref1} ZHANG Mingyu,Review of Researches on Urban Light Pollution in China.
    [A]Zhaoming Gongcheng Xuebao,2019.
    DOI:10.3969/j.issn.1004-440X.2019.05.006
    \bibitem{ref2} F. Falchi, P. Cinzano, D. Duriscoe, C.C.M. Kyba, C.D. Elvidge, K. Baugh, B.A. Portnov, N.A. Rybnikova, R. Furgoni. 
    The new world atlas of artificial night sky brightness. 
    Sci. Adv., 2 (2016), pp. 1-26, 10.1126/sciadv.1600377.
    \bibitem{ref3} Tongyu Wang, Naoko Kaida, Kosuke Kaida,
    Effects of outdoor artificial light at night on human health and behavior: A literature review,
    Environmental Pollution,
    Volume 323,
    2023,
    121321,
    ISSN 0269-7491,
    https://doi.org/10.1016/j.envpol.2023.121321.
    \bibitem{ref4} XIONG Ruiyu, CHEN Zheng. Review of the Impact of Artificial Light Pollution at Night on One Health and Key Thresholds[J]. 
    Chinese Landscape Architecture, 2023, 39(2):32-37. DOI:10.19775/j.cla.2023.02.0032.

\end{thebibliography}